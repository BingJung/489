\documentclass{article}

\title{Project 2 for PSYC 489}
\author{Bingjun Guo (bingjun3)}
\begin{document}
\maketitle
\section*{Part 1}
\subsection*{(1)}
\hspace{\parindent}(a) results are:
\[a_1 = 171.5, a_2 = 171.5, a_3 = 170.5\]
For the first two iteration, $a_1 = 0.5, a_2 = 0.5, a_3 = 1.5$, and then
$a_1 = 1.5, a_2 = 1.5, a_3 = 0.5$. Then the smallest input among the 3
nodes will be 2, which is larger than the largest activation, 1.5.
Thus, all nodes will keep increasing since their inputs will be constantly 
larger than their activations.\\

(b) results are:
\[a_1 = -169.5, a_2 = -170.5, a_3 = -170.5\]
After the first iteration, $a_1 = -0.5, a_2 = 0.5, a_3 = 0.5$; After the 
second iteration, $a_1 = 0.5, a_2 = -0.5, a_3 = -0.5$; After the third
iteratoin, $a_1 = -1.5, a_2 = -0.5, a_3 = -0.5$, and since then, it's
clear that sums of inputs and thus activations will drop being negative
with an increasing rate because they have been all negative.\\

(c) results are:
\[a_1 = -340.5, a_2 = -341.0, a_3 = -341.0\]
After the first iteration, $a_1$ becomes 0 and both $a_2$ and $a_3$ are -0.5.
Since all nodes are negative, activations will drop with an increasing rate.

\subsection*{(2)}
\hspace{\parindent}(a) results are:
\[a_1 = 1, a_2 = 1, a_3 = 1\]
Since the activation is bounded above this time, after the two initially
activated nodes activate the rest one, all nodes will hold being
activated and stays 1.\\

(b) results are:
\[a_1 = 1, a_2 = 1, a_3 = 1\]
The initially activated node will activate the rest 2, and all 3 stays 
activated 1 after that.\\

(c) results are:
\[a_1 = 0, a_2 = 0, a_3 = 0\]
After the first iteration, activations of all units will become 0 since
all of them are initially less than 0.75. Activations are bounded below
as well, unlike the case in (1).

\section*{Part 2}
results are:\\
(a1) \[a = 0.9999990463256836\]
(a2) \[a = -56.6650390625\]
(a3) \[a = 0.9990234375\]
(a4) \[a = -0.8984375\]
(a5) \[a = 0.992584228515625\]
(b1) \[a = -0.9914032911155202\]
(c1) \[a = -0.81626756145152\]
answers to the questions:
\subsection*{(1)}
Activation of $(a1)$ converges to 1 while $(a2)$ becomes negative and gets a large negative 
value. Because for each iteration of $(a1)$, $\gamma\cdot n_i = 0.75 < 1$,
which means that $a_1$ will keep increasing but never exceed 1 with such input. Meanwhile, 
for $(a2)$, $\gamma\cdot n_i = 2.5$, and $|(1-2.5)|=1.5 > 1$, which will lead 
to explosion since the absolute value of $\Delta a$ will only be larger and 
larger.

\subsection*{(2)}
Activation of $(a3)$ converges to 1 again. $a_1 = \gamma\cdot n_i = 1.5$, and 
$|(1-1.5)|=0.5 < 1$. With $a_i^t = a_i^{t-1} + 1.5\cdot (1-a_i^{t-1}) = 1.5 - 
0.5\cdot a_i^{t-1}$, $a_i$ will get closer to 1, and thus $0.5\cdot a_i$ 
will get closer to 0.5, making $1.5 - 0.5\cdot a_i$ closer to 1. As 
this logic keeps being consistent, $a_i$ is getting closer to 1.

\subsection*{(3)}
I think what happens on $(a5)$ can be considered as gradient vanishing, 
in which case $a_i$ is too close to 1 and thus $(1-a_i)$ is too small, 
making the same input ineffective. The reason why $(a5)$ suffers from this
is that the first 10 iteration has already make $a_i$ for $(a5)$ too close
to 1, leave the latter inputs ineffective. However, for $(a4)$, input
switches to negative after only 2 iterations, which successfully help $(a4)$
escape from the issue.

\subsection*{(4)}
Result in $(b1)$ becomes negative instead. In the first 10 iterations of $(b1)$, 
$\Delta a_i = 0.5\cdot (1 - a_i) - 0.1\cdot a_i = 0.5 - 0.6\cdot a_i$, and
thus $a_i^t = 0.5 + 0.4\cdot a_i^{t-1}$, making $a_i$ slowly increasing towards
1 like it's in $(a5)$. However, after the input of $(b1)$ switches to -1,
$\Delta a_i = -0.5\cdot (1 - a_i) - 0.1\cdot a_i = 0.4\cdot a_i - 0.5$, and
thus $a_i^t = 1.4\cdot a_i^{t-1} - 0.5$. Since after only 10 iterations in $(b1)$,
$0.4\cdot a_i < 0.5$, $a_i$ will keep decreasing, become negative, and go 
on decreasing.

\subsection*{(5)}
They will keep decreasing. For $(a4)$, $\Delta a_i = 0.5\cdot(a_i-1)$, hence
$a_i^t = 1.5\cdot a_i^{t-1} - 0.5$. For $(b1)$, as is stated previously, 
$a_i^t = 1.4\cdot a_i^{t-1} - 0.5$. In both cases, as soon as $(a_i)$ becomes
negative, it will go on decreasing faster in exponential manners.

\subsection*{(6)}
$(c1)$ will avoid the negative explosion and converge to a value with infinite
iterations taking -1 as input. When $a_i$ and input are both negative, $(b1)$ 
will be unbounded and explode. However, coefficient applied to the input for 
$(c1)$, will be $(1+a_i)$ instead, which makes the rule bounded when $-1<a_i< 0$
and thus avoid the explosion happening to $(b1)$.

\end{document}